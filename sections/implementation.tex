\chapter{Implementation} \label{chap:implementation}

The aim of this chapter is to present the implementation details of our system.
Each explanation is accompanied, when required, by code snippets to further illustrate the rationale behind our design choices.
For the full implementation details we refer to Appendix \ref{chap:app-code}.
The implementation we will cover is the one the results presented in Chapter \ref{chap:evaluation} are based on.
As a consequence, and in order to stress-test the system, we replace real sensors with synthetic data generators in the client package first described in \S\ref{sec:clients}.
All the different components introduced in Chapter \ref{chap:architecture} are packaged in sets of Docker containers.
Large fleets of concurrent users are then simulated using Docker's standalone clusters and mounted using \texttt{docker-machine}.

The rest of the chapter is structured as follows.
In \S\ref{sec:server-implementation} and \S\ref{sec:client-implementation} we cover the implementation details of the server-side and the client-side component respectively.
Re-using Docker-specific nomenclature, we will refer to logical isolated functionalities packaged in a container as \emph{services}.
Sets of services working collaboratively will be gathered and deployed as a \emph{component} using \texttt{docker-compose}.
Collections (or clusters) of replicated components form \emph{swarms} using \texttt{docker-swarm}.
Lastly, in \S\ref{sec:deployment} we cover how each service, component and swarm is deployed.

\section{Server Implementation} \label{sec:server-implementation}

We rely on the original \sgxspark implementation, and we only modify it to support a different in-enclave code deployment path, so that the \texttt{.jar} archive is available inside the enclaves and the shared memory. 
To do so, we include our newly added module in the project's \texttt{pom.xml} compilation file, in the worker and driver initialization scripts and in the enclave generation \texttt{Makefile} so that code is loaded in the enclave at compilation time.
The before-mentioned module contains two HRV processing algorithms and a benchmarking one: \texttt{SDNN}, \texttt{HRVBands}, and \texttt{Identity}.
The \texttt{SDNN} algorithm computes the standard deviation of NN (RR) intervals in a rolling basis, generating one output per 10 seconds worth of samples.
The \texttt{HRVBands}~\cite{Shaffer2017} performs a Discrete Fourier Transform (DFT) of 10 seconds worth of samples, and computes the power of the low frequency and high frequency components, together with their ratio.
Lastly, the \texttt{Identity} algorithm copies the input to the output file.
It is used to provide a baseline on the overhead the system is introducing.
All the application code is implemented in \textsc{Spark}'s binding for the \textsc{Scala} programming language~\cite{scala-language}. 
We choose this particular binding since it is the only one supported by \sgxspark's compiler.
To be usable inside \sgx enclaves, applications must adhere to the RDD~\cite{rdd-programming-guide} and DStreams~\cite{ZahariaDStreams2012} API.

The particular implementation of these algorithms relies on basic Spark Streaming operators, and their corresponding Scala counterparts.
For instance, the filesystem interface is monitored using DStream's \texttt{textFileStream} method as exposed in Listing~\ref{code:text-file-stream}. There, variables and code are initialized (lines 3-10) then the input directory is monitored (line 12) and the computation and output are assigned. Note that data is not processed until the Streaming Context is started via the \texttt{start()} method (line 22).
\begin{lstlisting}[language=Scala,caption={Snippet illustrating \texttt{textFileStream} functionality.},label=code:text-file-stream]
def main(args: Array[String]) {
      // Initialize variables and Spark Contexts
      val wdwSize = 10
      val numClients = args(0).toInt
      val conf = new SparkConf().setAppName("HRV Toolbox - SDNN")          
      val sc = new SparkContext(conf)                                    
      val ssc = new StreamingContext(sc, Duration(10000))                 
                                                                          
      // Initialize the Estimate Class
      val estimateHRV = new EstimateHRVBands(wdwSize)
      
      // Monitor Input Directory                           
      val dataStreamVec = for (i <- 1 to numClients) yield ssc.textFileStream("csem/src/main/resources/csv/"+i.toString+"/
")                                                                                                                     
      // Estimate
      val hrvBandVec = for (dS <- dataStreamVec) yield estimateHRV.estimate(dS)
      // Save as Output
      for (i <- 1 to numClients) {
        hrvBandVec(i-1).saveAsTextFiles("csem/src/main/resources/results/" + i.toString + "/sdnn")
      }
      // Start Stream Processing
      ssc.start()
      ssc.awaitTermination()
  }
\end{lstlisting}
Lastly, the implementations of \texttt{SDNN}, \texttt{HRVBands}, and \texttt{Identity} can be found in Listings~\ref{code:sdnn},~\ref{code:hrvbands}, and~\ref{code:identity} respectively.

\section{Client Implementation} \label{sec:client-implementation}

Clients correspond to body-sensors strapped to the body of a user working collaboratively with a smart gateway, sending gathered data to the cloud-based component. 
In a real deployment, the sensor service in the client component would be an actual sensor (\textit{e.g.} HR band, smartwatch, or optical HR monitoring device) and the gateway would be implemented in a, for instance, \textsc{Raspberry Pi}.
For evaluation purposes (see Chapter~\ref{chap:evaluation}), these services are simulated and virtualized.
This way, we can simulate a situation where multiple clients concurrently use the platform without the hardware burden.
As firstly introduced in \S\ref{sec:clients}, our implementation decouples the client component into into five different services (see Figure~\ref{fig:system-architecture}, right). 
\begin{enumerate}
    \item The \texttt{sensor} service is a \textsc{Python} script that generates \texttt{sample\_rate} random samples per second, where a sample is an RR interval and its timestamp, and publishes the information to the \texttt{artificial-data} topic in a \textsc{MQTT}~\cite{mqtt-protocol} broker located in the gateway. Listing~\ref{code:sensor-data-gen} presents the part of the \texttt{sensor} source code where artificial data is generated and published. The full source code is available in Listing~\ref{code:sensor}. Note the particularity of the sample generation procedure. First, \texttt{sample\_rate} random points in a $[0, 1)$ uniform distribution are generated (lines 18, 21). Then, current timestamps are inferred from them (line 24) and intervals are computed as successive differences (line 25). These intervals are, in fact, the RR intervals.
\begin{lstlisting}[language=Python,caption={Snippet illustrating the artificial data generation in the \texttt{sensor} service.},label=code:sensor-data-gen]
def loop(sample_rate):
    """Endless Sample Generation

    This module is the endless loop that generates <sample_rate>
    new samples per second and publishes them to a MQTT topic.

    Parameters
    ----------
    sample_rate : float
        How many samples are generated per second
    """
    killer = Killer()
    global CL_ID
    past_ts = time.time()
    while 1:
        value = ""
        # First we generate <sample_rate> timestamps in one second
        tstamps = [i for i in random.uniform(low=time.time(),
                                             high=time.time() + 1,
                                             size=(sample_rate))]
        tstamps.sort()
        # Second we compute the intervals between them
        for i in tstamps:
            tmp = datetime.fromtimestamp(i).strftime('%Y-%m-%d %H:%M:%S')
            value += tmp+','+str(int((i - past_ts) * 1e6))+'\n'
            past_ts = i
        publish.single("artificial-data-{}".format(CL_ID), value)
        time.sleep(1)
        if killer.kill_now:
            print("Exiting gracefully!")
            break
\end{lstlisting}
    \item The \texttt{eclipse-mqtt} service is the \texttt{mosquitto}~\cite{mqtt-eclipse} implementation of the \textsc{MQTT} broker. We rely on the public Docker image at Docker Hub~\cite{mqtt-image}. It handles the samples generated by \texttt{sensor} and delivers them to the \texttt{mqtt-subscriber} when the latter subscribes to the \texttt{artificial-data} topic.
    \item The \texttt{mqtt-subscriber} service is a \textsc{Python} script that subscribes to the \texttt{artificial-data} \textsc{MQTT} topic and whenever it gathers \texttt{FILE\_LINES} (line 26) samples it generates a \texttt{.csv} file at a specified location. The key functionality of the service, sample gathering and file generation, is illustrated in Listing~\ref{code:mqtt-sub-short}. The full service code is available in Listing~\ref{code:mqtt-sub}.
\begin{lstlisting}[language=Python,caption={Snippet of the data gathering and file generation in the \texttt{mqtt-subscriber} service.},label=code:mqtt-sub-short]
def on_message(mosq, obj, msg):
    """Callback method for the mosquitto client.

    Whenever the MQTT client receives a new sample it triggers this method.
    When the current count reaches the threshold, a new csv file is generated
    and stored in the local data directory.

    Parameters
    ----------
    Parameters are those of the standard implementation of the MQTT client
    for Python.
    """
    global count
    global curr_str
    global FILE_LINES
    global CL_ID
    global LOCAL_DATA_DIR
    global killer
    print("Received message")
    dcd_msg = msg.payload.decode("utf-8").split("\n")
    whole_msg = [i.split(",") for i in dcd_msg if i != ""] 
    for tmp in whole_msg:
        d = datetime.strptime(tmp[0], '%Y-%m-%d %H:%M:%S')
        time_stamp = time.mktime(d.timetuple())
        rr = tmp[1]
        if count == FILE_LINES:
            global TS
            print("It took: {} s".format(time.time() - TS))
            curr_str += "{},{}".format(time_stamp, rr)
            ts = int(time.time()*1e3)
            filename = "{}{}_".format(LOCAL_DATA_DIR, CL_ID)+str(ts)+".csv"
            try:
                with open(filename, "w+") as f:
                    f.write(curr_str)
            except FileNotFoundError as e:
                if killer.kill_now:
                    mosq.disconnect()
                    print("Exiting gracefully!")
                    break
            count = 0
            curr_str = ""
        else:
            curr_str += "{},{}\n".format(time_stamp, rr)
            count += 1
    else:
        if killer.kill_now:
            mosq.disconnect()
            print("Exiting gracefully!")
\end{lstlisting}
    \item The \texttt{producer} service is a \textsc{Python} script that monitors a local directory and, whenever a new file is stored, sends it to the remote filesystem. The monitoring functionality is depicted in Listing~\ref{code:producer-short} and the full source code is available in Listing~\ref{code:producer}. We keep track of the current files in the directory with a dictionary (lines 175 and 180) and we compute the difference between the pre and the post (line 181). We then transfer the new files over SFTP (line 188).
\begin{lstlisting}[language=Python,caption={Snippet illustrating the local directory monitoring in the \texttt{producer} service.},label=code:producer-short]
def monitor(ssh, sftp, *args):
    """Sending daemon

    This method starts an infinite loop that every <SEND_PERIOD> looks for
    newly added data files and copies them to a remote directory.

    Parameters
    ----------
    ssh : paramiko.SSHClient
        SSH Client connected to the remote host.
    sftp : paramiko.SFTPClient
        SFTP Client connected to the remote host.
    """
    global LOCAL_DATA_DIR
    path_to_watch = LOCAL_DATA_DIR
    before = dict([(f, None) for f in os.listdir(path_to_watch)])
    global SEND_PERIOD
    killer = Killer()
    while 1:
        time.sleep(SEND_PERIOD)
        after = dict([(f, None) for f in os.listdir(path_to_watch)])
        added = [f for f in after if f not in before]
        if added:
            for f in added:
                local_file = LOCAL_DATA_DIR + str(f)
                global REMOTE_DATA_DIR
                remote_file = REMOTE_DATA_DIR + "/" + str(f)
                sftp.put(local_file, remote_file)
                os.remove(local_file)
                # If there are a lot of files, not checking inside also leads
                # to error with code 137
                if killer.kill_now:
                    print("Exiting gracefully!")
                    break
        before = after
        if killer.kill_now:
            print("Exiting gracefully!")
            break
    for d in os.listdir(LOCAL_DATA_DIR):
        tmp_file = LOCAL_DATA_DIR + str(d)
        os.remove(tmp_file)
\end{lstlisting}
    \item Similarly, the \texttt{consumer} service is a \textsc{Python} script that monitors the remote filesystem and, whenever a result file is stored, fetches it to the local result directory. The monitoring functionality is illustrated in Listing~\ref{code:consumer-short} and the whole source code is available in Listing~\ref{code:consumer}. It is, in essence, symmetric to the one presented for the \texttt{producer} service. The only difference is that, since \texttt{sftp.get()} does not support recursive invokations (\textit{e.g.} \texttt{-R} flags), specific care must be taken to fetch all the different files placed in the different directories.
\begin{lstlisting}[language=Python,caption={Snippet illustrating the remote filesystem monitoring in the \texttt{consumer} service.},label=code:consumer-short]
def monitor(ssh, sftp):
    """Fetching daemon

    This method starts an infinite loop that every <FETCH_PERIOD> looks for
    newly added directories and copies them to a local result directory.

    Parameters
    ----------
    ssh : paramiko.SSHClient
        SSH Client connected to the remote host.
    sftp : paramiko.SFTPClient
        SFTP Client connected to the remote host.
    """
    global REMOTE_RESULT_DIR
    global FETCH_PERIOD
    global LOCAL_RESULT_DIR
    #print("CONSUMER: user -> {}".format(getpass.getuser()))
    path_to_watch = REMOTE_RESULT_DIR
    before = dict([(f, None) for f in sftp.listdir(path_to_watch)])
    killer = Killer()
    while 1:
        time.sleep(FETCH_PERIOD)
        after = dict([(f, None) for f in sftp.listdir(path_to_watch)])
        added = [f for f in after if f not in before]
        if added:
            for f in added:
                local_dir = LOCAL_RESULT_DIR + str(f) + "/"
                remote_dir = REMOTE_RESULT_DIR + "/" + str(f) + "/"
                try:
                    os.mkdir(local_dir)
                    get_all(sftp, remote_dir, local_dir)
                except FileExistsError as e:
                    print("Fetching a file that already exists!")
        before = after
        if killer.kill_now:
            print("Exiting gracefully!")
            break
    for d in os.listdir(LOCAL_RESULT_DIR):
        tmp_dir = LOCAL_RESULT_DIR + d
        for f in os.listdir(tmp_dir):
            tmp_file = tmp_dir + "/" + f
            os.remove(tmp_file)
        os.rmdir(tmp_dir)
\end{lstlisting}
\end{enumerate}
All the services implemented in Python, amount to a total of $888$ Lines of Code (LoC). 
As introduced before, for the service deployment we rely on Docker.
The corresponding \texttt{Dockerfile}s and \texttt{docker-compose} are presented in \S\ref{sec:deployment}.
Furthermore, the deployment of each service is also wrapped in a \texttt{Dockerfile} and included in an image.
Lastly, we remark that the communication between the client and the server happens via SSH/SFTP to ensure transport layer security when transferring user's data.

\section{Deployment} \label{sec:deployment}

To ease scalability and reproducibility of both server and client, deployment is orchestrated by a single script detached from both execution environments. 
This is done through a combination of \texttt{bash} scripts that act as entry points for the \texttt{Dockerfile} and \texttt{docker-compose} file.
We differentiate between the server deployment (\S\ref{sec:deployment:server}), the client deployment (\S\ref{sec:deployment:client}) and the overall deployment and evaluation (\S\ref{sec:deployment:all}).

\subsection{Server Execution Deployment} \label{sec:deployment:server}

Specifying the remote location; the \textsc{SGX-Spark} engine, the streaming algorithm, and the filesystem interface are initialized. 
Depending on whether we want to enable enclaves or not (\texttt{SGX\_MODE}, line 24), a set of scripts (lines 25 and 26) are executed in the remote server using \texttt{ssh.exec\_command()}.
The script then remains passive until a \texttt{SIGTERM} signal is captured, and all processes are killed (lines 41-49).
A snippet of the main functionality is attached in Listing~\ref{code:deployment:server-short} and the full source code is available in Listing~\ref{sec:deployment:server}.
Note that the code is intended to be launched from a location different from the server (which we leave untrusted).

\begin{lstlisting}[language=Python,caption={Main method of the Server-Side Deployment Script.},label=code:deployment:server-short]
def main(ssh, sftp):
    """Main execution method

    This method contains the main deployment of UC1. It basically starts each
    and every process, launches the benchmarking, and cleans everything when
    execution has finished.

    Notes
    -----
    All the sleeps are placed due to experienced errors race conditions. As a
    consequence their arguments are completely empirical.

    Arguments
    ---------
    ssh : paramiko.SSHClient
        SSH Client to the remote server.
    sftp : paramiko.SFTPClient
        SFTP Client to the remote server.
    """
    global ALGORITHM
    global SGX_MODE
    global NUM_CLIENTS
    algo_name = "./launch-csem-{}.sh {}".format(ALGORITHM, NUM_CLIENTS)
    if SGX_MODE:
        script_list = ["./master.sh", "./worker.sh", "./worker-enclave.sh",
                       "./driver-enclave.sh", algo_name]
    else:
        script_list = ["./master.sh", "./worker.sh", algo_name]
    for script in script_list:
        _in, _out, _err = ssh.exec_command("cd sgx-spark; {} &".format(script))
        time.sleep(3)
    tmp_command = "dstat -l --nocolor --noheaders 10 > cpu_load.csv"
    _, _out, __ = ssh.exec_command("cd sgx-spark; {} &".format(tmp_command))
    _out.readlines()
    killer = Killer()
    while 1:
        time.sleep(1)
        if killer.kill_now:
            print("Killing Gracefully!")
            break
    for script in script_list:
        _, _out, __ = ssh.exec_command("kill $(pgrep -f '{}')".format(script))
        _out.readlines()
    _, _out, __ = ssh.exec_command("kill $(pgrep -f 'java')")
    _out.readlines()
    _, _out, __ = ssh.exec_command("kill $(pgrep -f 'dstat')")
    _out.readlines()
    _, _out, __ = ssh.exec_command("rm /dev/shm/*")
    _out.readlines()
    return 0
\end{lstlisting}

\subsection{Client Execution Deployment} \label{sec:deployment:client}

The client deployment procedure is more complicated than the server's one since we support executions of variable number of clients, which are instantiated at start-up time.
In short, when the number of clients is set, the platform starts a standalone Docker cluster (or \emph{swarm}).
However, once started with the scalability tests, we recalled that the number of clients a cluster whose network relies on Docker's default \texttt{bridge} adapter can support was too small.
At 20 clients, the engine ran out of local IP addresses and executions would systematically fail.
As a consequence, we decided to virtualize a local standalone cluster using \texttt{docker-machine}.
This latter functionality enables the deployment of several Docker engines in a single computing instance.
To do so, a virtual machine must be started for each engine, and special care must be taken to set up the network, so that all containers are reachable from the others.

Even though the full script is rather long, given its complexity and how crucial it is to the system's performance, we attach it entirely in Listing~\ref{code:deploy-client}.
We firstly deploy a name discovery service based on the \textsc{Consul}~\cite{consul-image} docker image, the aim of which is to keep track of all the generated virtual machines, and assign an IP address to each one (lines 27-36).
Secondly, we start $\lceil \texttt{NUM\_CLIENTS} / 20 \rceil$ virtual machines and register them with the name discovery service (lines 40-60).
These set of virtual machines will form our Docker Swarm, as a consequence, it is important to elect one (first by default) as our \texttt{swarm-master}.
Note that, in order to speed up the start-up time, the images for all the client services are compressed and pre-loaded to the virtual machines (lines 62-66).
In third place, we create the overlay network (lines 70-71).
Then, the local directories for each client in each virtual machine must be mounted and mirrored outside the virtual machine so that we can process the results.
This is done using SSHF in lines 76-85.
Lastly, each client is deployed using the specific \texttt{docker-compose} file (see Listing~\ref{code:client-compose}).
\begin{lstlisting}[language=sh,caption={Client Cluster Deployment Script.},label=code:deploy-client]
#!/bin/bash

set -a

# Predefined enviroment variables
source docker-env.env

# Pull MQTT image
docker-compose pull mqtt

# Max Containers per Host
MAX_CONTAINERS=2

# Input Parameters
SAMPLE_RATE=${1}
NUM_CLIENTS=${2}
FILE_LINES=${3}

NUM_HOSTS=$(( $(( $NUM_CLIENTS / $MAX_CONTAINERS )) + 1))

# To introduce an abstraction layer, even if there is no real need for an
# overlay network (the bridge itself suffices) we create an external machine
# anyways. This may be an overkill for very small executions but is introduced
# to preserve scalability transparent to the user.

# We first create the Consul discovery service (key/value store)
docker-machine create \
    -d virtualbox \
    --virtualbox-boot2docker-url ~/tmp/boot2docker.iso \
    mh-keystore
eval "$(docker-machine env mh-keystore)"
docker run -d \
  --name consul \
  -p "8500:8500" \
  -h "consul" \
  consul agent -server -bootstrap -client "0.0.0.0"

# We then create the Swarm cluster the first node (id == 1) will be the swarm
# manager.
for HOST_ID in $(seq 1 ${NUM_HOSTS}); 
do
    if [ $HOST_ID == 1 ]; then
        docker-machine create \
            -d virtualbox \
            --virtualbox-boot2docker-url ~/tmp/boot2docker.iso \
            --swarm --swarm-master \
            --swarm-discovery="consul://$(docker-machine ip mh-keystore):8500" \
            --engine-opt="cluster-store=consul://$(docker-machine ip mh-keystore):8500" \
            --engine-opt="cluster-advertise=eth1:2376" \
            "sgx-csem-host-$HOST_ID"
    else
        docker-machine create \
            -d virtualbox \
            --virtualbox-boot2docker-url ~/tmp/boot2docker.iso \
            --swarm \
            --swarm-discovery="consul://$(docker-machine ip mh-keystore):8500" \
            --engine-opt="cluster-store=consul://$(docker-machine ip mh-keystore):8500" \
            --engine-opt="cluster-advertise=eth1:2376" \
            "sgx-csem-host-$HOST_ID"
    fi
    eval $(docker-machine env "sgx-csem-host-$HOST_ID")
    docker load -i ~/tmp/consumer.tar
    docker load -i ~/tmp/deployment.tar
    docker load -i ~/tmp/fake-sensor.tar
    docker load -i ~/tmp/mqtt-sub.tar
    docker load -i ~/tmp/producer.tar
done

# We now create the overlay network
eval $(docker-machine env --swarm sgx-csem-host-1)
docker network create --driver overlay --subnet=10.0.0.0/16 sgx-csem-net

for CL_ID in $(seq 1 ${NUM_CLIENTS});
do
    # Prepare environment
    H_ID=$(( $(( $CL_ID / $MAX_CONTAINERS )) + 1))
    mkdir -p ${LOCAL_DATA_DIR}${CL_ID}
    mkdir -p ${LOCAL_RESULT_DIR}${CL_ID}
    eval $(docker-machine env "sgx-csem-host-$H_ID")
    docker-machine ssh "sgx-csem-host-$H_ID" mkdir -p "results/${CL_ID}"
    docker-machine mount "sgx-csem-host-$H_ID":"/home/docker/results/$CL_ID" \
        ${LOCAL_RESULT_DIR}${CL_ID}
    docker-machine ssh "sgx-csem-host-$H_ID" mkdir -p "data/${CL_ID}"
    docker-machine mount "sgx-csem-host-$H_ID":"/home/docker/data/$CL_ID" \
        ${LOCAL_DATA_DIR}${CL_ID}

    # Deploy containers (Add a -d at the end for detach mode)
    cat ${COMPOSE_FILE} | envsubst | docker-compose -f - -p \
        "${PROJECT_NAME}-${CL_ID}" up &
    sleep 5
done

# Return to default environment before exiting
eval $(docker-machine env -u)
\end{lstlisting}

\subsection{Deployment} \label{sec:deployment:all}

The main entry point for the platform's deployment is included in Listing~\ref{code:deployment:whole}.
It first loads a set of environment variables and parses the input (lines 6,8-12), it then deploys the client cluster as detailed in \S\ref{sec:deployment:client}, and it lastly deploys the server container as detailed in \S\ref{sec:deployment:server}.
\begin{lstlisting}[language=sh,caption={Main entry point for a single execution.},label=code:deployment:whole]
#!/bin/bash

set -a

# Predefined enviroment variables
source execution.env

ALGORITHM=${1}
SGX_MODE=${2}
SAMPLE_RATE=${3}
NUM_CLIENTS=${4}
FILE_LINES=${5}

# Deploy Client-Side container
cd ${CLIENT_DIR}
bash ${CLIENT_DIR}/deploy-client.sh ${SAMPLE_RATE} ${NUM_CLIENTS} ${FILE_LINES}
cd ${WDIR}

# Deploy Server-Side Containers
cat ${COMPOSE_FILE} | envsubst | docker-compose -f - -p "${PROJECT_NAME}" up &
\end{lstlisting}

Additionally, a \textsc{Python} script wraps the launcher presented in Listing~\ref{code:deployment:whole} in order to conveniently orchestrate execution batches for evaluation purposes.
The same script (attached in Listing~\ref{code:deployment:experiments}) also includes the queries to obtain the required metrics for the different benchmarking scenarios that we will present in Chapter~\ref{chap:evaluation}.
