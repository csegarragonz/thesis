\chapter{Implementation Code Snippets} \label{chap:app-code}

\section{Server-Side Algorithms}

\begin{lstlisting}[language=Scala,caption={Implementation of the \texttt{SDNN} algorithm.},label=code:sdnn]
/** Estimation of the Standard Deviation between NN intervals.
 *                                                                
 *  @constructor create a new estimation suite.
 *  @param wdwSize duration in seconds of the window in which we compute
 *  the standard deviation.
 */                                                                   
class EstimateSDNN(wdwSize: Int) { 

  /** Estimation of the SDNN with a fixed window in streaming mode
   *
   *  @return returns a DStream containing a tuple per line containing the time
   *  block/interval (starting from zero) and the SDNN value.
   */
  def sdnn(dataStream: DStream[String]): DStream[(String, Double)] = {
    val operatorByKey = new StreamOperators(wdwSize)
    operatorByKey.sumSquareByKey(dataStream)
      .join(operatorByKey.cardinalByKey(dataStream))
      .map(udfDivide)
      .join(operatorByKey.avgByKey(dataStream))
      .map(udfSqrtDif)
      .map(udfTimesWindow)
  }

}
\end{lstlisting}

\begin{lstlisting}[language=Scala,caption={Implementation of the \texttt{HRVBands} algorithm.},label=code:hrvbands]
package org.apache.spark.csem.utils

import org.apache.spark.rdd.RDD
import org.apache.spark.streaming.dstream.DStream
import org.apache.spark.csem.utils.StreamOperators

import scala.math.{pow, cos, sin, Pi, floor}

/** Set of signal processing operators defined on (K, V), (String, Float) 
 *  DStreams. Currently the methods included in this suite are: DFT,
 *
 *  @constructor create a new operator suite. 
 *  @param wdwSize duration in seconds of the window in which we compute
 *  the standard deviation.
 */
class StreamSignalProcessing(wdwSize: Int) extends Serializable {

  /** Preprocesses the input data stream (which contains lines from the csv
   *  input file) to the following format, (k, v) where:
   *  - k: is the 10 second block identifier (key)
   *  - v: iterable with all the rr's from that block indexed.
   *
   *  @return A data stream with the specified format.
   */
  def preProcess(dataStream: DStream[String]):
    DStream[(String, Iterable[(Double, Int)], Int)] = {
      dataStream
        .map(line => (floor(line.split(",")(0).toDouble/10).toInt.toString,
          line.split(",")(1).toDouble))
        .groupByKey()
        .map(line => (line._1, line._2.zipWithIndex, line._2.size))
  }

  /** Compute the squared module of the Discrete Fourier Transform of the 
   *  samples contained in the DStream.
   *
   *  @return returns a DStream containing a tuple per line containing the time
   *  block/interval (starting from zero) and the sum of all the values
   *  in that same interval.
   */
  def modDFT2(dataStream: DStream[String]):
    DStream[(String, Double, Double, Double)] = { 

    // Auxiliary methods to compute the squared modulus of the DFT
    def cosDFT(line: (String, Iterable[(Double, Int)], Int), f_index: Int):
      Double = {
      line._2
        .map(x => x._1*cos(2*Pi*x._2*f_index/line._3))
        .reduce((a,b) => (a+b))
    }
    def sinDFT(line: (String, Iterable[(Double, Int)], Int), f_index: Int):
      Double = {
      line._2
        .map(x => x._1*sin(2*Pi*x._2*f_index/line._3))
        .reduce((a,b) => (a+b))
    }

    // Compute the Square Modulus of the DFT
    def modDFT(value: (Double, Int),
      line: (String, Iterable[(Double, Int)], Int)): (Double, Int) = {
        (pow(cosDFT(line, value._2), 2) + pow(sinDFT(line, value._2), 2),
          value._2)
    }

    // Predicate to know if a value belongs to the High Freq. Component
    def isHighFreq(value: (Double, Int), cardinal: Int): Boolean = {
      val thr_low = (0.15 * cardinal).toInt
      val thr_high = (0.4 * cardinal).toInt
      value._2 match {
        case x if thr_low until thr_high contains x => true
        case _ => false
      }
    }

    // Predicate to know if a value belongs to the Low Freq. Component
    def isLowFreq(value: (Double, Int), cardinal: Int): Boolean = {
      val thr_low = (0.04 * cardinal).toInt
      val thr_high = (0.15 * cardinal).toInt
      value._2 match {
        case x if thr_low until thr_high contains x => true
        case _ => false
      }
    }

    val tmp = preProcess(dataStream)
      .map( l => (l._1, l._2.map( list => modDFT(list, l) ), l._3) )

    // High Frequency Component
    val highFreq = tmp
      .map(line => (line._1, line._2
        .filter( x => isHighFreq(x, line._3) )
        .map( x => x._1 )
        .foldLeft(0.toDouble)( (a,b) => (a + b) ), line._3))
      .map( line => (line._1, line._2 / pow(line._3, 2) * 2) )

    // Low Frequency Component
    val lowFreq = tmp
      .map(line => (line._1, line._2
        .filter(x => isLowFreq(x, line._3))
        .map( x => x._1 )
        .foldLeft(0.toDouble)( (a,b) => (a + b) ), line._3))
      .map( line => (line._1, line._2 / pow(line._3, 2) * 2) )
    lowFreq
      .join(highFreq)
      .map( x => (x._1, x._2._1, x._2._2, x._2._1 / x._2._2) )
  }
}


/** Estimation of the frequency bands given a HRV sequence.
 *
 *  @constructor create a new estimation suite.
 *  @param wdwSize duration in seconds of the window in which we estimate
 *  the frequency bands.
 */
class EstimateHRVBands(wdwSize: Int) {
  /** Estimation of the SDNN with a fixed window in streaming mode
   *
   *  @return returns a DStream containing a tuple per line containing the time
   *  block/interval (starting from zero) and the SDNN value.
   */
  def estimate(dataStream: DStream[String]):
      DStream[(String, Double, Double, Double)] = {
    val processingByKey = new StreamSignalProcessing(wdwSize)
    processingByKey.modDFT2(dataStream)
  }
}
\end{lstlisting}

\begin{lstlisting}[language=Scala,caption={Implementation of the \texttt{Identity} algorithm.},label=code:identity]
/** Main application/class that launches the identity application.                                                [64/180]
 *                                                                                                                        
 *  @note We implement the identity application that                                                 
 *  basically does nothing. This way we will be able to measure easily the                                                
 *  system's overhead.
 */
object Identity extends Logging {

  def main(args: Array[String]) {

    // Initialize Variables and Spark Context
    val numClients = args(0).toInt
    val conf = new SparkConf().setAppName("HRV Toolbox - Identity")
    val sc = new SparkContext(conf)
    val ssc = new StreamingContext(sc, Duration(10000))

    // Map Output to Input
    val dataStreamVec = for (i <- 1 to numClients) yield ssc.textFileStream("csem/src/main/resources/csv/"+i.toString+"/
")
    for (i <- 1 to numClients) {
    dataStreamVec(i-1).saveAsTextFiles("csem/src/main/resources/results/" + i.toString + "/identity")
    }

    ssc.start()
    ssc.awaitTermination()
  }
}
\end{lstlisting}

\section{Client-Side Services Source Code}

\begin{lstlisting}[language=Python,caption={Implementation of the \texttt{sensor} service.},label=code:sensor]
"""Fake MQTT Generator

This module emulates a sensor that publishes data to a MQTT topic with a given
sample rate.
"""
// Method names are those required to be overridden by Paho MQTT
import paho.mqtt.publish as publish 
import signal
import sys
import time
from datetime import datetime
from numpy import random


// SIGTERM Signal Handler
class Killer:
  kill_now = False
  def __init__(self):
    signal.signal(signal.SIGTERM, self.exit_gracefully)

  def exit_gracefully(self, signum, frame):
    self.kill_now = True

    
// Callback to be executed at service start up time
def run(sample_rate):
    if sample_rate == 0:
        raise ZeroDivisionError("Can not set a 0 sample rate")
    loop(int(sample_rate))


def loop(sample_rate):
    """Endless Sample Generation

    This module is the endless loop that generates <sample_rate>
    new samples per second and publishes them to a MQTT topic.

    Parameters
    ----------
    sample_rate : float
        How many samples are generated per second
    """
    killer = Killer()
    global CL_ID
    past_ts = time.time()
    while 1:
        value = ""
        # First we generate <sample_rate> timestamps in one second
        tstamps = [i for i in random.uniform(low=time.time(),
                                             high=time.time() + 1,
                                             size=(sample_rate))]
        tstamps.sort()
        # Second we compute the intervals between them
        for i in tstamps:
            tmp = datetime.fromtimestamp(i).strftime('%Y-%m-%d %H:%M:%S')
            value += tmp+','+str(int((i - past_ts) * 1e6))+'\n'
            past_ts = i
        publish.single("artificial-data-{}".format(CL_ID), value)
        time.sleep(1)
        if killer.kill_now:
            print("Exitting gracefully!")
            break


if __name__ == "__main__":
    """Standalone entry point"""
    if len(sys.argv) < 2:
        raise TypeError("No Sample Rate provided")
    CL_ID = sys.argv[2]
    run(sys.argv[1])
\end{lstlisting}

\begin{lstlisting}[language=Python,caption={Implementation of the \texttt{mqtt-subscriber} service.},label=code:mqtt-sub]
"""MQTT Subscriber and CSV Generator

This module subscribes to the data topic in a MQTT queue and, whenever it
stacks a given amount of samples it generates a new csv file.

Attributes
----------
count : int
    Global counter.
curr_str : str
    Global string accumulator.
"""
import paho.mqtt.client as mqtt
from datetime import datetime
import time
import signal
import sys
import os


count = 0
curr_str = ""


class Killer:
  kill_now = False
  def __init__(self):
    signal.signal(signal.SIGTERM, self.exit_gracefully)

  def exit_gracefully(self, signum, frame):
    self.kill_now = True


def on_message(mosq, obj, msg):
    """Callback method for the mosquitto client.

    Whenever the MQTT client receives a new sample it triggers this method.
    When the current count reaches the threshold, a new csv file is generated
    and stored in the local data directory.

    Parameters
    ----------
    Parameters are those of the standard implementation of the MQTT client
    for Python.
    """
    global count
    global curr_str
    global FILE_LINES
    global CL_ID
    global LOCAL_DATA_DIR
    global killer
    print("Received message")
    dcd_msg = msg.payload.decode("utf-8").split("\n")
    whole_msg = [i.split(",") for i in dcd_msg if i != ""] 
    for tmp in whole_msg:
        d = datetime.strptime(tmp[0], '%Y-%m-%d %H:%M:%S')
        time_stamp = time.mktime(d.timetuple())
        rr = tmp[1]
        if count == FILE_LINES:
            global TS
            print("It took: {} s".format(time.time() - TS))
            curr_str += "{},{}".format(time_stamp, rr)
            ts = int(time.time()*1e3)
            filename = "{}{}_".format(LOCAL_DATA_DIR, CL_ID)+str(ts)+".csv"
            try:
                with open(filename, "w+") as f:
                    f.write(curr_str)
            except FileNotFoundError as e:
                if killer.kill_now:
                    mosq.disconnect()
                    print("Exitting gracefully!")
                    break
            count = 0
            curr_str = ""
        else:
            curr_str += "{},{}\n".format(time_stamp, rr)
            count += 1
    else:
        if killer.kill_now:
            mosq.disconnect()
            print("Exitting gracefully!")


def start_mqtt():
    """Start the MQTT client"""
    global CL_ID
    mqttc = mqtt.Client()
    mqttc.on_message = on_message
    mqttc.connect("localhost", port=1883)
    mqttc.subscribe("artificial-data-{}".format(CL_ID))
    mqttc.loop_forever()


if __name__ == "__main__":
    """Standalone execution entry point"""
    try:
        FILE_LINES = int(sys.argv[1])
    except IndexError as e:
        print("Have not provided a FILE_LINES value!")
        sys.exit(1)
    killer = Killer()
    CL_ID = sys.argv[2]
    LOCAL_DATA_DIR = sys.argv[3]
    start_mqtt()
    TS = time.time()
\end{lstlisting}

\begin{lstlisting}[language=Python,caption={Implementation of the \texttt{producer} service.},label=code:producer]
"""CSV Producer

This module monitors a local data directory and perodically sends the newly
added csv files to a remote directory in order to be processed.

Attributes
----------
"""
import os
import time
import signal
import sys
from paramiko import SSHClient, SSHConfig, ProxyCommand, SFTPClient
from paramiko.util import log_to_file
from server_connect import server_connect


class Killer:
  kill_now = False
  def __init__(self):
    signal.signal(signal.SIGTERM, self.exit_gracefully)

  def exit_gracefully(self, signum, frame):
    self.kill_now = True


def monitor(ssh, sftp, *args):
    """Sending daemon

    This method starts an infinite loop that every <SEND_PERIOD> looks for
    newly added data files and copies them to a remote directory.

    Parameters
    ----------
    ssh : paramiko.SSHClient
        SSH Client connected to the remote host.
    sftp : paramiko.SFTPClient
        SFTP Client connected to the remote host.
    """
    global LOCAL_DATA_DIR
    path_to_watch = LOCAL_DATA_DIR
    before = dict([(f, None) for f in os.listdir(path_to_watch)])
    global SEND_PERIOD
    killer = Killer()
    while 1:
        time.sleep(SEND_PERIOD)
        after = dict([(f, None) for f in os.listdir(path_to_watch)])
        added = [f for f in after if f not in before]
        if added:
            for f in added:
                local_file = LOCAL_DATA_DIR + str(f)
                global REMOTE_DATA_DIR
                remote_file = REMOTE_DATA_DIR + "/" + str(f)
                sftp.put(local_file, remote_file)
                os.remove(local_file)
                # If there are a lot of files, not checking inside also leads
                # to error with code 137
                if killer.kill_now:
                    print("Exitting gracefully!")
                    break
        before = after
        if killer.kill_now:
            print("Exitting gracefully!")
            break
    for d in os.listdir(LOCAL_DATA_DIR):
        tmp_file = LOCAL_DATA_DIR + str(d)
        os.remove(tmp_file)


if __name__ == "__main__":
    """Standalone entry point

    The scripts is executed as a subprocess and as a consequence in standalone
    mode. This is done to prevent it from blocking the rest of the execution
    and to prevent from working with Python's threading library.
    """
    ssh, sftp = server_connect()
    LOCAL_DATA_DIR = sys.argv[1]
    REMOTE_DATA_DIR = sys.argv[2]
    SEND_PERIOD = int(sys.argv[3])
    sftp.mkdir(REMOTE_DATA_DIR)
    monitor(ssh, sftp)
\end{lstlisting}

\begin{lstlisting}[language=Python,caption={Implementation of the \texttt{consumer} service.},label=code:consumer]
"""Result Consumer Daemon

This module periodically fetches data from a remote directory containing the
results of the computation and copies them over SFTP to a local directory.

Attributes
----------
FETCH_PERIOD : int
    How often does the module fetch the newly generated directories.
"""
import os
import getpass
import sys
import time
import signal
from paramiko import SSHClient, SSHConfig, ProxyCommand, SFTPClient
from paramiko.util import log_to_file
from server_connect import server_connect


class Killer:
  kill_now = False
  def __init__(self):
    signal.signal(signal.SIGTERM, self.exit_gracefully)

  def exit_gracefully(self, signum, frame):
    self.kill_now = True


def get_all(sftp, remote_dir, local_dir):
    """Copy a whole directory

    The aim of this module is to emulate the -r option in scp. Note that this
    is not recursive and only works in a folder with a known and given
    structure.

    Parameters
    ----------
    sftp : paramiko.SFTPClient
        SFTP Client connected to the remote host.
    remote_dir : str
        Path of the directory we are copying from.
    local_dir : str
        Path of the directory we are copying to.
    """
    for f in sftp.listdir(remote_dir):
        remote_file = remote_dir + str(f)
        if (not f.startswith('.')) and (not f.startswith('_')):
            local_file = local_dir + str(f)
            sftp.get(remote_file, local_file)


def monitor(ssh, sftp):
    """Fetching daemon

    This method starts an infinite loop that every <FETCH_PERIOD> looks for
    newly added directories and copies them to a local result directory.

    Parameters
    ----------
    ssh : paramiko.SSHClient
        SSH Client connected to the remote host.
    sftp : paramiko.SFTPClient
        SFTP Client connected to the remote host.
    """
    global REMOTE_RESULT_DIR
    global FETCH_PERIOD
    global LOCAL_RESULT_DIR
    #print("CONSUMER: user -> {}".format(getpass.getuser()))
    path_to_watch = REMOTE_RESULT_DIR
    before = dict([(f, None) for f in sftp.listdir(path_to_watch)])
    killer = Killer()
    while 1:
        time.sleep(FETCH_PERIOD)
        after = dict([(f, None) for f in sftp.listdir(path_to_watch)])
        added = [f for f in after if f not in before]
        if added:
            for f in added:
                local_dir = LOCAL_RESULT_DIR + str(f) + "/"
                remote_dir = REMOTE_RESULT_DIR + "/" + str(f) + "/"
                try:
                    os.mkdir(local_dir)
                    get_all(sftp, remote_dir, local_dir)
                except FileExistsError as e:
                    print("Fetching a file that already exists!")
        before = after
        if killer.kill_now:
            print("Exitting gracefully!")
            break
    for d in os.listdir(LOCAL_RESULT_DIR):
        tmp_dir = LOCAL_RESULT_DIR + d
        for f in os.listdir(tmp_dir):
            tmp_file = tmp_dir + "/" + f
            os.remove(tmp_file)
        os.rmdir(tmp_dir)


if __name__ == "__main__":
    """Standalone entry point

    The scripts is executed as a subprocess and as a consequence in standalone
    mode. This is done to prevent it from blocking the rest of the execution
    and to prevent myself from working with Python's threading library.
    """
    ssh, sftp = server_connect()
    LOCAL_RESULT_DIR = sys.argv[1]
    REMOTE_RESULT_DIR = sys.argv[2]
    sftp.mkdir(REMOTE_RESULT_DIR)
    FETCH_PERIOD = int(sys.argv[3])
    monitor(ssh, sftp)
\end{lstlisting}
\section{Deployment Scripts}

\begin{lstlisting}[language=Python,caption={Server-Side Deployment Script.},label=code:deployment:server]
#!/usr/bin/python3
"""Remote Side Standalone MedSpark Launcher

This module implements the remote-side, standalone, launcher for MedSpark.
"""
import os
import sys
import time
import signal
from server_connect import server_connect


class Killer:
  kill_now = False
  def __init__(self):
    signal.signal(signal.SIGTERM, self.exit_gracefully)

  def exit_gracefully(self, signum, frame):
    self.kill_now = True


def main(ssh, sftp):
    """Main execution method

    This method contains the main deployment of UC1. It basically starts each
    and every process, launches the benchmarking, and cleans everything when
    execution has finished.

    Notes
    -----
    All the sleeps are placed due to experienced errors race conditions. As a
    consequence their arguments are completely empirical.

    Arguments
    ---------
    ssh : paramiko.SSHClient
        SSH Client to the remote server.
    sftp : paramiko.SFTPClient
        SFTP Client to the remote server.
    """
    global ALGORITHM
    global SGX_MODE
    global NUM_CLIENTS
    algo_name = "./launch-csem-{}.sh {}".format(ALGORITHM, NUM_CLIENTS)
    if SGX_MODE:
        script_list = ["./master.sh", "./worker.sh", "./worker-enclave.sh",
                       "./driver-enclave.sh", algo_name]
    else:
        script_list = ["./master.sh", "./worker.sh", algo_name]
    for script in script_list:
        _in, _out, _err = ssh.exec_command("cd sgx-spark; {} &".format(script))
        time.sleep(3)
    tmp_command = "dstat -l --nocolor --noheaders 10 > cpu_load.csv"
    _, _out, __ = ssh.exec_command("cd sgx-spark; {} &".format(tmp_command))
    _out.readlines()
    killer = Killer()
    while 1:
        time.sleep(1)
        if killer.kill_now:
            print("Killing Gracefully!")
            break
    for script in script_list:
        _, _out, __ = ssh.exec_command("kill $(pgrep -f '{}')".format(script))
        _out.readlines()
    _, _out, __ = ssh.exec_command("kill $(pgrep -f 'java')")
    _out.readlines()
    _, _out, __ = ssh.exec_command("kill $(pgrep -f 'dstat')")
    _out.readlines()
    _, _out, __ = ssh.exec_command("rm /dev/shm/*")
    _out.readlines()
    return 0


if __name__ == "__main__":
    """Entry point for standalone executions

    This entry point contains the command line argument parsing.
    """
    ssh, sftp = server_connect("/home/user/logfile.log")
    ALGORITHM = sys.argv[1]
    SGX_MODE = int(sys.argv[2])
    NUM_CLIENTS = int(sys.argv[3])
    main(ssh, sftp)
\end{lstlisting}

\begin{lstlisting}[language=docker-compose,caption={Client Docker Compose Script.},label=code:client-compose]
version: '3'

services:
    consumer:
        build: ./services/consumer/
        container_name: "sgx-csem-client-consumer-${CL_ID}"
        image: consumer:latest
        command: "${CONTAINER_RESULT_DIR} ${REMOTE_RESULT_DIR}/${CL_ID} ${FETCH_PERIOD}"
        user: "${myUID}:${myGID}"
        volumes: 
            - "/home/docker/results/${CL_ID}:${CONTAINER_RESULT_DIR}"
              #    - "${LOCAL_RESULT_DIR}${CL_ID}:${CONTAINER_RESULT_DIR}"
    producer:
        build: ./services/producer/
        container_name: "sgx-csem-client-producer-${CL_ID}"
        image: producer:latest
        command: "${CONTAINER_DATA_DIR} ${REMOTE_DATA_DIR}/${CL_ID} ${SEND_PERIOD}"
        volumes: 
            - "/home/docker/data/${CL_ID}:${CONTAINER_DATA_DIR}"
              #- "${LOCAL_DATA_DIR}${CL_ID}:${CONTAINER_DATA_DIR}"
    mqtt-sub:
        build: ./services/mqtt_sub/
        container_name: "sgx-csem-client-mqtt-sub-${CL_ID}"
        image: mqtt-sub:latest
        network_mode: "sgx-csem-net"
        user: "${myUID}:${myGID}"
        depends_on: 
            - "mqtt"
        volumes: 
            - "/home/docker/data/${CL_ID}:${CONTAINER_DATA_DIR}"
              #- "${LOCAL_DATA_DIR}${CL_ID}:${CONTAINER_DATA_DIR}"
        command: "${FILE_LINES} ${CL_ID} ${CONTAINER_DATA_DIR}"
    fake-sensor:
        build: ./services/fake-sensor/
        container_name: "sgx-csem-client-fake-sensor-${CL_ID}"
        image: fake-sensor:latest
        network_mode: "sgx-csem-net"
        command: "${SAMPLE_RATE} ${CL_ID}"
        depends_on: 
            - "mqtt"
    mqtt:
        image: eclipse-mosquitto
        container_name: "sgx-csem-mqtt-${CL_ID}"
        network_mode: "sgx-csem-net"
        logging:
            driver: none
\end{lstlisting}

\begin{lstlisting}[language=Python,caption={Benchmarking and Experiment Deployment Script.},label=code:deployment:experiments]
#!/usr/bin/python3
"""Benchmarking and Evaluation launcher.

This module includes all the diferent benchmarking scenarions considered in the
project. It also deploys all of them in a sequential fashion.
"""
import itertools
import sys
import os
import time
import subprocess
from datetime import datetime
import requests
import glob
import shutil
from server_connect import server_connect


def query_batch_processing(filedir):
    """Query Spark's REST API 
    This method queries sparks API for the batch processing time. It firsts
    launches a port forwarding daemon in order to be able to perform the
    request and then queries the information. Note that this is done only
    once at the end of the execution (right before killing it).
    """
    proc = subprocess.Popen("python3 port_forward.py".split(" "))
    time.sleep(2)
    app_id = requests.get('http://localhost:4040/api/v1/applications/').json()
    app_id = app_id[0]['id']
    req = 'http://localhost:4040/api/v1/applications/{}/jobs/'.format(app_id)
    r = requests.get(req).json()
    time_format = "%Y-%m-%dT%H:%M:%S.%f"
    times = [ datetime.strptime(r[i]['completionTime'][:-3],
                                time_format).timestamp() -
              datetime.strptime(r[i]['submissionTime'][:-3],
                                time_format).timestamp()
              for i in reversed(range(len(r))) if 'completionTime' in r[i] ]
    proc.kill()
    filename = filedir + "batch_processing_times.csv"
    with open(filename, "w+") as f:
        for num, val in enumerate(times):
            if num == 0:
                f.write("{} {}".format(num, val))
            else:
                f.write("\n{} {}".format(num, val))
    return 0


def query_energy(filedir, elapsed_time = 0, energy_ini = 0):
    """ Query UniNe's Powe Control System

    The idea is to compute consumed energy vs runtime so this are the values
    that we will pack. For further reference below are the values to query
    for other things.
    pdu_val   |   variable measured
       1      |   Voltage AC rms (V)
       2      |   Current AC rms (A)
       8      |   Total energy active (kWh)
      10      |   Resettable energy (?) active (kWh)
    DISCLAIMER: The current result is in kWh!
    """
    port_fw = "python3 port_forward.py"
    port_fw += " -lp 8080 -ru powercontrol.maas -rp 80 -rh hoernli-2"
    proc = subprocess.Popen("{}".format(port_fw).split(" "))
    time.sleep(2)
    req = 'http://localhost:8080/statusjsn.js?components=16384'
    # In this case we choose value 8 -> Energy
    r = requests.get(req).json()['sensor_values'][1]['values'][3][8]['v']
    # Obtain teh shit
    # subprocess.call("kill $(pgrep -f '{}')".format(port_fw), shell=True)
    proc.kill()
    if energy_ini:
        filename = filedir + "energy_consumption.csv"
        with open(filename, "w+") as f:
            f.write("{} {}\n".format(elapsed_time, float(r) - energy_ini))
    else:
        return float(r)


def query_memory(filedir):
    ssh, sftp = server_connect("./logfile.log")
    local_file = filedir + "cpu_load.csv"
    remote_file = "/home/ubuntu/sgx-spark/cpu_load.csv"
    sftp.get(remote_file, local_file)
    sftp.remove(remote_file)
    ssh.close()
    sftp.close()


# TODO: This will eventually have to change when we containerize the client
def query_active_clients(filedir, total_clients):
    ssh, sftp = server_connect("./logfile.log")
    filename = filedir + "clients.csv"
    remote_dir = "/home/ubuntu/sgx-spark/csem/src/main/resources/csv/"
    active_clients = len(set([l.split('_')[0] for l in sftp.listdir(remote_dir)]))
    with open(filename, "w+") as f:
        f.write("Active clients:\t {}\n".format(active_clients))
        f.write("Total clients:\t {}\n".format(total_clients))
    ssh.close()
    sftp.close()


def main():
    """Launcher.

    This method defines all the diferent benchmarking environments, the
    parameters they take and deploys each and every execution.
    """
    # General variables
    #n_versions = 1
    #sr = [1]
    #batch_number = 200
    #batch_duration = 10

    # Evaluation Variables
    # Test:
    #n_clients = [100]
    #algorithms = ["sdnn"]
    #mode = [1]
    # Client Scalability:
    #n_clients = [1, 50, 250, 500]
    #algorithms = ["sdnn", "identity", "hrvbands"]
    #mode = [1, 0]
    # Maximal Throughput Many Files
    n_versions = 1
    sr = [100, 200, 300, 400, 500]
    batch_number = 20
    batch_duration = 10
    n_clients = [1]
    algorithms = ["sdnn"]
    mode = [0]

    for elem in itertools.product(*[algorithms, mode, sr]):
        for version in range(n_versions):
            for client in n_clients:
                print("Starting execution with the following parameters:")
                print("\t - Replica: {} of {}".format(version + 1, n_versions))
                print("\t - Number of clients: {}".format(client))
                print("\t - Algorithm: {}".format(elem[0]))
                print("\t - SGX Enabled: {}".format(elem[1]))
                print("\t - Sample Rate: {}".format(elem[2]))
                # The dynamic file size is set so that one file is generated
                # roughly every 10 seconds, so simply sample_rate * 10 lines.
                fl = elem[2] * 10
                subprocess.call("./deploy-single-evaluation.sh {} {} {} {} {}".format(
                                 elem[0], elem[1], elem[2], client, fl).split(" "))
                tmp = "/{}_ncli_{}_sgx_{}_sr_{}_v{}/".format(elem[0], client,
                                                             elem[1], elem[2],
                                                             version)
                filedir = OUTPUT_DIR + tmp
                os.makedirs(filedir, exist_ok=True)
                try:
                    energy_ini = query_energy(filedir)
                    elapsed_time = batch_number*batch_duration
                    time.sleep(elapsed_time)
                    query_batch_processing(filedir)
                    query_energy(filedir, elapsed_time, energy_ini)
                    query_memory(filedir)
                    query_active_clients(filedir, client)
                except Exception as e:
                    print("An exception occured during the data grepping!")
                    print(e)
                finally:
                    subprocess.call("./kill-execution.sh")


if __name__ == "__main__":
    OUTPUT_DIR = sys.argv[1]
    main()
\end{lstlisting}

\chapter{Evaluation Code Snippets} \label{chap:app:evaluation}

\begin{lstlisting}[language=Python,caption={Python Script to Set Up a Port Forwarding Daemon.},label=code:evaluation:port-forward]
#!/usr/bin/python3
"""SSH Tunnel Daemon

This module starts a daemon that establishes a SSH tunnel from the local box
to the remote one. This is done to be able to query Spark's UI. It is
essentially a local port forwarding.
"""
import os
import sys
import argparse
import socket
from paramiko import SSHClient, SSHConfig, ProxyCommand
from paramiko.util import log_to_file
from forward import forward_tunnel


def run(**kwargs):
    """Main method

    Starts a subprocess that SSH's to the remote host and establishes the
    tunnel/port forwarding. In short, you will see at localhost:local_port
    the contents found at remote_url:remote_port as seen by remote_host.

    Attributes
    ----------
    local_port : int
        Local port towards which the connection is forwarded.
    remote_url : str
        Remote url you are accessing from the remote host.
    remote_port : int
        Port you will be accessing the remote url through.
    remote_host : str
        Host through which you are establishing the tunnel. Note that this
        host should appear in your .ssh/config and, due to particularities
        of the implementation, it must include a ProxyCommand to be used.
    """
    ssh = SSHClient()
    ssh_config = SSHConfig()
    ssh.load_system_host_keys()
    with open(os.path.expanduser("~/.ssh/config")) as f:
        ssh_config.parse(f)
    try:
        details = ssh_config.lookup(kwargs['remote_host'])
    except Exception as e:
        logger.error("The remote host {} is not listed in your .ssh/config" +
                     " file!".format(kwargs['remote_host']))
    log_to_file("logfile.log")
    ssh.connect(hostname=details['hostname'], username=details['user'],
                sock=ProxyCommand(details['proxycommand']))
    forward_tunnel(kwargs['local_port'], kwargs['remote_url'],
                   kwargs['rem_port'], ssh.get_transport())


if __name__ == "__main__":
    parser = argparse.ArgumentParser()
    parser.add_argument("-lp", "--local_port", help="Local forwarded port",
                        type=int, nargs='?', default=4040, const=4040)
    parser.add_argument("-ru", "--remote_url", help="End URL (from remote)",
                        type=str, nargs='?', default='localhost',
                        const='localhost')
    parser.add_argument("-rp", "--rem_port", help="Remote used port", type=int,
                        nargs='?', default=4040, const=4040)
    parser.add_argument("-rh", "--remote_host", help="Remote host", type=str,
                        nargs='?', default="eiger-9", const="eiger-9")
    args = parser.parse_args()
    run(**vars(args))
\end{lstlisting}

\begin{lstlisting}[language=Python,caption={Python Script to Set Up a SSH Tunnel.},label=code:evaluation:forward-tunnel]
#!/usr/bin/env python

# Copyright (C) 2003-2007  Robey Pointer <robeypointer@gmail.com>
#
# This file is part of paramiko.
#
# Paramiko is free software; you can redistribute it and/or modify it under the
# terms of the GNU Lesser General Public License as published by the Free
# Software Foundation; either version 2.1 of the License, or (at your option)
# any later version.
#
# Paramiko is distributed in the hope that it will be useful, but WITHOUT ANY
# WARRANTY; without even the implied warranty of MERCHANTABILITY or FITNESS FOR
# A PARTICULAR PURPOSE.  See the GNU Lesser General Public License for more
# details.
#
# You should have received a copy of the GNU Lesser General Public License
# along with Paramiko; if not, write to the Free Software Foundation, Inc.,
# 59 Temple Place, Suite 330, Boston, MA  02111-1307  USA.

"""
Sample script showing how to do local port forwarding over paramiko.

This script connects to the requested SSH server and sets up local port
forwarding (the openssh -L option) from a local port through a tunneled
connection to a destination reachable from the SSH server machine.
"""

import os
import socket
import select

try:
    import SocketServer
except ImportError:
    import socketserver as SocketServer

import sys
from optparse import OptionParser

import paramiko


SSH_PORT = 22
DEFAULT_PORT = 4000


class ForwardServer(SocketServer.ThreadingTCPServer):
    daemon_threads = True
    allow_reuse_address = True


class Handler(SocketServer.BaseRequestHandler):
    def handle(self):
        try:
            chan = self.ssh_transport.open_channel(
                "direct-tcpip",
                (self.chain_host, self.chain_port),
                self.request.getpeername(),
            )
        except Exception as e:
            return
        if chan is None:
            return

        while True:
            r, w, x = select.select([self.request, chan], [], [])
            if self.request in r:
                data = self.request.recv(1024)
                if len(data) == 0:
                    break
                chan.send(data)
            if chan in r:
                data = chan.recv(1024)
                if len(data) == 0:
                    break
                self.request.send(data)

        peername = self.request.getpeername()
        chan.close()
        self.request.close()


def forward_tunnel(local_port, remote_host, remote_port, transport):
    # this is a little convoluted, but lets me configure things for the Handler
    # object.  (SocketServer doesn't give Handlers any way to access the outer
    # server normally.)
    class SubHander(Handler):
        chain_host = remote_host
        chain_port = remote_port
        ssh_transport = transport

    ForwardServer(("", local_port), SubHander).serve_forever()
\end{lstlisting}
