\chapter{Conclusions and Lessons Learnt} \label{chap:conclusion}

In this thesis, we have presented a proof of concept of a streaming platform that grants executions on remote, untrusted, servers or clouds with data and code confidentiality and integrity. 
We provide end-to-end protection transparently to the developer since we run unmodified \textsc{Apache Spark} applications inside \textsc{Intel SGX}'s enclaves.

Our design easily scales to different types of data generators, data streams, or even processing algorithms.
It only relies on \sgxspark, a stream processing framework.
However, this dependency could also be overcomed since the server component in our architecture is also pluggable and modular.

We have quantified the impact on overall system performance when protecting health sensitive data from an untrusted cloud provider.
More precisely, when performing an HRV analysis, for files smaller than 4 MB it introduces a x4-5 slow-down when compared to vanilla \textsc{Apache Spark} both in batch and streaming execution mode. 
A good part of this slow-down though, is due to the collaborative JVM structure adopted in \sgxspark.
For a matter of fact, a slow-down of only x1.5-2 is introduced when moving from \sgxspark without enclaves to with enclaves.
Regardless, these results are already competitive when compared to other S.o.A privacy-preserving processing engines~\cite{Zheng2017}. 
Therefore, we consider our platform to be mature enough to be introduced in a production environment, since it complies with current data protection regulations whilst still maintaining a reasonable performance, and keeping the costs to use the cloud infrastructure reasonable.

