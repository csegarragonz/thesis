%% ENGLISH VERSION ------------------------------------------------------------
\addcontentsline{toc}{chapter}{Abstract}
\topskip0pt
\vspace*{2.00cm}
\begin{center}
%    \large
%    UNIVERSITAT POLIT\`ECNICA DE CATALUNYA (UPC) 
%
%    \normalsize
%    Facultat de Matem\`atiques i Estad\'istica (FME)
%
%    Escola T\`ecnica Superior d'Enginyeria de Telecomunicaci\'o de Barcelona (ETSETB)
%
%    Centre de Formaci\'o Interdisciplin\`aria Superior (CFIS)
%
%    \vspace{0.5cm}
%
%    \large
%    SWISS CENTER FOR ELECTRONICS AND MICROTECHNOLOGY (CSEM)
%    \normalsize
%    Embedded Software Group - Systems Division
%    
%    \vspace{0.5cm}

    \LARGE
    \textit{\textbf{Abstract}} \label{sec:abstract}

    \vspace{0.5cm}

    \large
    \textbf{\projName: Using Trusted Execution Environments for Secure Stream Processing of Medical Data}

    by \textsc{Carlos Segarra Gonz\'alez}
\end{center}

\vspace{0.5cm}

\normalsize
Processing sensitive data, specially medical data produced by body sensors, on third-party untrusted clouds is particularly challenging without compromising the privacy of the users generating it. Typically, these sensors generate large quantities of continuous data in a streaming fashion. Such vast amount of information must be processed efficiently and securely, even under strong adversarial models. The recent introduction in the mass-market of consumer-grade processors with Trusted Execution Environments (TEEs), such as Intel SGX, paves the way to implement solutions that overcome less flexible approaches, such as those atop homomorphic encryption. 
    
This Bachelor Thesis presents \projName, a secure streaming processing system built on top of Intel SGX. To showcase the viability of this approach, we use it with a system specifically fitted for medical data. We design and fully implement a prototype system that we evaluate with several realistic datasets. Our experimental results show that \projName introduces a reduced overhead to vanilla Spark while offering strong additional protection guarantees under powerful attackers and threat models.

\vspace{0.5cm}

\textbf{Keywords:} TEE, Trusted Hardware, Stream Processing, Intel SGX, Spark

\vfill
\pagebreak

%% CATALAN VERSION ------------------------------------------------------------
\topskip0pt
\vspace*{2cm}
\begin{center}
%    \large
%    UNIVERSITAT POLIT\`ECNICA DE CATALUNYA (UPC) 
%
%    \normalsize
%    Facultat de Matem\`atiques i Estad\'istica (FME)
%
%    Escola T\`ecnica Superior d'Enginyeria de Telecomunicaci\'o de Barcelona (ETSETB)
%
%    Centre de Formaci\'o Interdisciplin\`aria Superior (CFIS)
%
%    \vspace{0.5cm}
%
%    \large
%    SWISS CENTER FOR ELECTRONICS AND MICROTECHNOLOGY (CSEM)
%    \normalsize
%    Embedded Software Group - Systems Division
%    
%    \vspace{0.5cm}
%
    \LARGE
    \textit{\textbf{Resum}} 

    \vspace{0.5cm}

    \large
    \textbf{\projName: Using Trusted Execution Environments for Secure Stream Processing of Medical Data}

    per \textsc{Carlos Segarra Gonz\'alez}
\end{center}

\vspace{0.5cm}

\normalsize
TRANSLATE TO CATALAN %TODO
%Processing sensitive data, specially medical data produced by body sensors, on third-party untrusted clouds is particularly challenging without compromising the privacy of the users generating it. Typically, these sensors generate large quantities of continuous data in a streaming fashion. Such vast amount of information must be processed efficiently and securely, even under strong adversarial models. The recent introduction in the mass-market of consumer-grade processors with Trusted Execution Environments (TEEs), such as Intel SGX, paves the way to implement solutions that overcome less flexible approaches, such as those atop homomorphic encryption. 

El processat de dades de car\`acter personal, especialment les d'or\'igen medical, en servidors remots al n\'uvol, \'es particularment complicat sense atemptar contra la privacitat dels usuaris que generen la informaci\'o.
Molt habitualment aquestes dades s\'on generades per sensors corporals que emeten un flux continu d'informaci\'o. Aquesta no nom\'es ha de ser processada de manera eficient, sin\'o tamb\'e de forma segura, fins i tot sota adversaris poderosos.
La recent introducci\'o al mercat de processadors amb Entorns d'Execuci\'o Segura (\textit{Trusted Execution Environments}), com ara Intel SGXX, faciliten la implementaci\'o de solucions 
    
This Bachelor Thesis presents \projName, a secure streaming processing system built on top of Intel SGX. To showcase the viability of this approach, we use it with a system specifically fitted for medical data. We design and fully implement a prototype system that we evaluate with several realistic datasets. Our experimental results show that \projName achieves modest overhead compared to vanilla Spark while offering additional protection guarantees under powerful attackers and threat models.

\vspace{0.5cm}

\textbf{Keywords:} TEE, Trusted Hardware, Stream Processing, Intel SGX, Spark

\vfill
