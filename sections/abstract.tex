%% ENGLISH VERSION ------------------------------------------------------------
\addcontentsline{toc}{chapter}{Abstract}
\topskip0pt
\vspace*{2.00cm}
\begin{center}
%    \large
%    UNIVERSITAT POLIT\`ECNICA DE CATALUNYA (UPC) 
%
%    \normalsize
%    Facultat de Matem\`atiques i Estad\'istica (FME)
%
%    Escola T\`ecnica Superior d'Enginyeria de Telecomunicaci\'o de Barcelona (ETSETB)
%
%    Centre de Formaci\'o Interdisciplin\`aria Superior (CFIS)
%
%    \vspace{0.5cm}
%
%    \large
%    SWISS CENTER FOR ELECTRONICS AND MICROTECHNOLOGY (CSEM)
%    \normalsize
%    Embedded Software Group - Systems Division
%    
%    \vspace{0.5cm}

    \LARGE
    \textit{\textbf{Abstract}} \label{sec:abstract}

    \vspace{0.5cm}

    \large
    \textbf{\projName: Using Trusted Execution Environments for Secure Stream Processing of Medical Data}

    by \textsc{Carlos Segarra Gonz\'alez}
\end{center}

\vspace{0.5cm}

\normalsize
Processing sensitive data, specially medical data produced by body sensors, on third-party untrusted clouds is particularly challenging without compromising the privacy of the users generating it. Typically, these sensors generate large quantities of continuous data in a streaming fashion. Such vast amount of information must be processed efficiently and securely, even under strong adversarial models. The recent introduction in the mass-market of consumer-grade processors with Trusted Execution Environments (TEEs), such as Intel SGX, paves the way to implement solutions that overcome less flexible approaches, such as those atop homomorphic encryption. 
    
This Bachelor Thesis presents \projName, a secure streaming processing system built on top of Intel SGX. To showcase the viability of this approach, we use it with a system specifically fitted for medical data. We design and fully implement a prototype system that we evaluate with several realistic datasets. Our experimental results show that \projName introduces a reduced overhead to vanilla Spark while offering strong additional protection guarantees under powerful attackers and threat models.

\vspace{0.5cm}

\textbf{Keywords:} TEE, Trusted Hardware, Stream Processing, Intel SGX, Spark

\vfill
\pagebreak

%% CATALAN VERSION ------------------------------------------------------------
\topskip0pt
\vspace*{2cm}
\begin{center}
%    \large
%    UNIVERSITAT POLIT\`ECNICA DE CATALUNYA (UPC) 
%
%    \normalsize
%    Facultat de Matem\`atiques i Estad\'istica (FME)
%
%    Escola T\`ecnica Superior d'Enginyeria de Telecomunicaci\'o de Barcelona (ETSETB)
%
%    Centre de Formaci\'o Interdisciplin\`aria Superior (CFIS)
%
%    \vspace{0.5cm}
%
%    \large
%    SWISS CENTER FOR ELECTRONICS AND MICROTECHNOLOGY (CSEM)
%    \normalsize
%    Embedded Software Group - Systems Division
%    
%    \vspace{0.5cm}
%
    \LARGE
    \textit{\textbf{Resum}} 

    \vspace{0.5cm}

    \large
    \textbf{\projName: Using Trusted Execution Environments for Secure Stream Processing of Medical Data}

    per \textsc{Carlos Segarra Gonz\'alez}
\end{center}

\vspace{0.5cm}

\normalsize

El processat de dades de car\`acter personal, especialment aquelles provinents de dominis m\`edics, en servidors remots al n\'uvol, \'es particularment delicat quan es vol preservar la privacitat dels usuaris que les generen.
Molt habitualment, aquestes dades provenen de petits sensors que l'usuari du posats i que emeten un flux continu de mesures.
Dit volum de mesures no nom\'es han de ser processades de manera eficient, sin\'o tamb\'e de manera segura, fins i tot sota hipot\`etics atacants amb acc\'es privilegiat a les m\`aquines al n\'uvol.
La recent introducci\'o al mercat de processadors amb Entorns d'Execuci\'o Segura (\textit{Trusted Execution Environments}), com ara Intel SGX, faciliten la implementaci\'o de solucions m\'es flexibles i lleugeres que les basades en esquemes de criptografia homom\`orfica.
    
Aquest Treball de Final de Grau presenta \projName, una plataforma de processament segur de fluxos que es basa en Intel SGX.
Per il{\tiny\raisebox{.9ex}{\textbullet}}lustrar la viabilitat de la plataforma, la usem en un entorn m\`edic.
En el decurs del treball, dissenyem i implementem un sistema prototip que evaluem amb jocs de dades reals.
Els nostres resultats experimentals mostren que \projName introdueix un discret ralentiment en comparaci\'o amb la implementaci\'o est\`andard de Spark, tot oferint un nivell de protecci\'o adicional per a les dades dels usuaris. 
%This Bachelor Thesis presents \projName, a secure streaming processing system built on top of Intel SGX. To showcase the viability of this approach, we use it with a system specifically fitted for medical data. We design and fully implement a prototype system that we evaluate with several realistic datasets. Our experimental results show that \projName achieves modest overhead compared to vanilla Spark while offering additional protection guarantees under powerful attackers and threat models.

\vspace{0.5cm}

\textbf{Paraules Clau:} Entorns d'Execuci\'o Segura, Computaci\'o Segura, Processament de Fluxos, Intel SGX, Spark

\vfill
