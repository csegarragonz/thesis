\chapter{Related Work} \label{chap:related-work}

In this Chapter we cover the related work that sets the current State-of-the-Art for privacy preserving stream processing engines of medical data.
Together with each piece of work, we provide a brief description and argue why it has or has not been considered for our project.
In \S\ref{sec:related:stream} we cover the most notable big data stream processing engines that relate with our system.
Note that stream processing is a big field and it is out of the scope of this work to provide a detailed survey.
In \S\ref{sec:related:privacy} we introduce relevant privacy-preserving processing engines: both stream and batch based.
Lastly, in \S\ref{sec:related:cardiac} we cover other approaches at cardiac data monitoring.

\section{Stream Processing Engines} \label{sec:related:stream}

Stream processing has recently attracted a lot of attention from academia and industry~\cite{Koliousis2016,Miao2017,Venkataraman2017}.
Apache Spark~\cite{Zaharia2012} is arguably the de-facto standard in this domain, by combining batch and stream processing with a unified API~\cite{ZahariaDStreams2012}.
Apache Spark SQL~\cite{Armbrust2015} allows to process structured data by integrating relational processing with Spark's functional programming style.
Structured streaming~\cite{Armbrust2018} leverages Spark SQL and it compares favorably against the discretized counterpart~\cite{ZahariaDStreams2012} in terms of performance. % approach used in this project. 
However, the former lacks security or privacy guarantees, and hence it was not considered.
Furthermore \sgxspark only has support for the \textit{Discretized Streams} API, and therefore we also rely on it.

\section{Privacy-Preserving Computation} \label{sec:related:privacy}

Opaque~\cite{Zheng2017} is a privacy-preserving distributed analytics system.
It leverages Spark SQL and Intel SGX enclaves to perform computations over encrypted Spark DataFrames.
In \texttt{encryption} mode, Opaque offers security guarantees similar to our system's.
However, \emph{(1)} the Spark master must be located in the client side, for it must be trusted. 
An scenario that does not fit in our multi-client setting.
And \emph{(2)}, it requires changes to the application code, and hence is not transparent to the user.
In \texttt{oblivious} mode, \emph{i.e.}, protecting against traffic pattern analysis attacks, it can be up to $46\times$ slower, a slow-down factor not tolerable for the real-time analytics in the scope of our system.

SecureStreams~\cite{Havet2017} is a reactive framework that exploits Intel SGX to define dataflow processing by pipelining several independent components. 
In order to use this framework, the user requires a good knowledge of the underlying implementation. 
Moreover, applications must be written in the \textsc{Lua} programming language, hindering its applicability to legacy systems or third-party programs.

\textsc{DPBSV}~\cite{Puthal2015} is a secure big data stream processing framework that focuses on securing data transmission from the sensors or clients to the \textit{Data Stream Manager (DSM)} or server. 
Its security model requires a \textit{Public Key Infrastructure (PKI)} and a dynamic prime number generation technique to synchronously update the keys. 
In spite of using trusted hardware on the DSM end for key generation and management, the server-side processes all the data in clear, making the framework not suitable for our security model. 

\textsc{TaLoS}~\cite{Aublin2017} is a \textit{Transport Layer Security (TLS)}~\cite{Dierks2008} library that terminates connections by maintaining sensitive information inside enclaves. 
The authors provide custom wrappers for the most common \textsc{TLS} API calls and securely store users and sessions' keys inside enclaves.
The embedding with a native Spark, or SGX-Spark, application is not transparent to the end user and hence why we discard it.

Lastly, homomorphic encryption~\cite{Gentry2009} does not rely on trusted execution environments and offers the promise of providing privacy-preserving computations over encrypted data.
This is, it generates an encrypted result from two encrypted operands, corresponding to the encrypted value of operating both operands in plain text.
The key feature of the scheme is that, at no point, neither the operands nor the result are processed in clear.
While several works analyzed the feasibility of homomorphic encryption schemes in cloud environments~\cite{Tetali2013,Stephen2016}, the performance of homomorphic operations~\cite{Gottel2018} is far from being pragmatic.

\section{Cardiac Monitoring Systems} \label{sec:related:cardiac}
% \vs{briefly discuss \cite{Puthal:2015:DES:2847604.2848495}}
% Talk about medical things
For the specific problem of HRV analysis, while periodic monitoring solutions exist~\cite{Renevey2018}, they are focused on embedded systems.
As such, since they off-load computation to third-party cloud services, these solutions simply overlook the privacy concerns that we consider.
Similarly, another solution~\cite{VanZaen2019} uses convolutional networks for the detection of arrhythmias.
However, the authors take no considerations with regard to data security and privacy.

%To the best of our knowledge, there are no privacy-preserving real-time streaming systems specifically designed for  medical and cardiac data. 
This work is one of the first attempts to building a privacy-preserving real-time streaming system specifically designed for  medical and cardiac data. 

Our system fills the existing research gap by proposing a system that leverages Intel SGX enclaves to compute such analytics over public untrusted clouds without changing the existing Scala-based source code.
