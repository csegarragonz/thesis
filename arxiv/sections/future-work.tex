\chapter{Future Work} \label{chap:future-work}

There are several dimensions along which the current project can be improved. 
We differentiate between possible improvements to the current architecture as presented in Chapter~\ref{chap:architecture}, and further research lines to enrich the privacy-preserving computing state of the art.

With regard to the project here presented, there are different issues that would need to be addressed in the coming work.
Firstly, the current \sgxspark implementation, as mentioned in Chapter~\ref{chap:evaluation}, does not yet have support for in-enclave streaming.
As a consequence, the estimated results should be validated once the Streaming API is supported by \sgxspark.
Secondly, results concerning client scalability should also be included in the evaluation chapter.
Following the deployment technique (standalone \texttt{Docker} cluster) indicated in \S\ref{sec:deployment}, we have managed to run experiments with hundreds of clients.
However, due to time constraints, we have not been able to provide an exhaustive assessment of the system's scalability, \textit{i.e.} how many clients it can run in parallel, neither of how the overhead introduced by \sgxspark might affect the number of clients our platform can handle simultaneously.
On the same lines, we would like to study the cost of deploying our system over public cloud infrastructures such as AWS Confidential Computing.
Lastly, we intend to deploy the platform in a real use case, this is, using real data produced by real users and streamed through a smart gateway.

On a more general note, we think that our threat model, as presented in \S\ref{sec:threat}, could be very much improved by securing the client package.
To do so, we envision the use of ARM TrustZone, widely available in low-power devices (\textit{e.g.}, Raspberry PI), to reduce the TCB in the client-side of the architecture whilst still leveraging on Trusted Execution Environments.
Finally, were TEEs to be used to secure the client package, a point-to-point, TEE-to-TEE, communication could be leveraged instead of relying on standard secure transfer protocols.
This way, data would never leave the TEE (to be included in a SFTP package and sent over the network), hence reducing the overall attack surface.
Even though generic TEE-to-TEE point-to-point communication protocols might not be mature enough yet (for generic TEEs on each endpoint), there are for instance solutions for enclave-to-enclave~\cite{Aublin2017} secure link establishment.
